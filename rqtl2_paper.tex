\documentclass[12pt,letterpaper]{article}

%%%%% packages
\usepackage[top=1in, bottom=1in, left=1in, right=1in]{geometry}
\usepackage{lscape,pdflscape}
\usepackage{alltt,amsmath,caption,color,epsfig,enumerate,enumitem,float,
  graphicx,hyperref,indentfirst,pdfpages,relsize,sectsty,setspace,subcaption,times}
\hypersetup{colorlinks, allcolors={black}}  % default link color
\hypersetup{pdfpagemode=UseNone} % don't show bookmarks on initial view
\setlength{\rightskip}{0pt plus 1fil} % makes ragged right
\usepackage[authoryear]{natbib}
\bibpunct{(}{)}{;}{a}{}{,}
%%%%%%%%%%%%%%

\allsectionsfont{\normalfont\sffamily\bfseries}
\subsectionfont{\normalfont\fontsize{12}{15}\sffamily\bfseries}

\newcommand{\LOD}{\text{LOD}}
\renewcommand{\figurename}{\textbf{Figure}}
\renewcommand{\thefigure}{\textbf{\arabic{figure}}}

\begin{document}

\setstretch{2.0}

\vspace*{8mm}
\begin{center}

\textbf{\Large R/qtl2: software for mapping quantitative trait loci
  with high-dimensional data and multi-parent populations}

\bigskip \bigskip \bigskip \bigskip

{\large Karl W. Broman$^{*,1}$, Daniel M. Gatti$^{**}$, Petr Simecek$^{**}$,
  Nicholas A. Furlotte$^{\S}$, \\
  Pjotr Prins$^{\dagger\dagger,\S\S}$, \'Saunak
  Sen$^{\ddagger\ddagger}$, Brian S. Yandell$^{\dagger,\ddagger}$,
  Gary A. Churchill$^{**}$}

\bigskip \bigskip

Departments of $^{*}$Biostatistics and Medical Informatics,
$^{\dagger}$Horticulture, and $^{\ddagger}$Statistics, University of
Wisconsin--Madison, Madison, Wisconsin 53706; $^{\S}$23andMe, Mountain
View, California 94043; $^{**}$The Jackson Laboratory, Bar Harbor, Maine 04609;
Departments of $^{\dagger\dagger}$Genetics, Genomics, and Informatics; and
$^{\ddagger\ddagger}$Preventive Medicine, University of
Tennessee Health Sciences Center, Memphis, Tennessee 38163;
$^{\S\S}$Center for Molecular Medicine, University Medical Center Utrecht, 3584CT Utrecht, The
Netherlands

\end{center}

%%% Add today's date
\def\todayISO{\leavevmode\hbox{\the\year-\twodigits\month-\twodigits\day}}
\def\twodigits#1{\ifnum#1<10 0\fi\the#1}

\vspace{3in} \hfill {\footnotesize \todayISO}
%%%%%%%%%%%%%%%%%

\newpage

\noindent \textbf{Running head:} R/qtl2 software


\bigskip \bigskip \bigskip

\noindent \textbf{Key words:} software, QTL,
multi-parent populations, MAGIC, Diversity Outbred mice, heterogeneous stock




\bigskip \bigskip \bigskip

\noindent \textbf{$^1$Corresponding author:}

\begin{tabular}{lll}
 \\
 \hspace{1cm} & \multicolumn{2}{l}{Karl W Broman} \\
 & \multicolumn{2}{l}{Department of Biostatistics and Medical Informatics} \\
 & \multicolumn{2}{l}{University of Wisconsin--Madison} \\
 & \multicolumn{2}{l}{2126 Genetics-Biotechnology Center} \\
 & \multicolumn{2}{l}{425 Henry Mall} \\
 & \multicolumn{2}{l}{Madison, WI 53706} \\
 \\
 & Phone: & 608--262--4633 \\
 & Email: & \verb|broman@wisc.edu|
\end{tabular}


\newpage

\section*{Abstract}

R/qtl2 is an interactive software environment for mapping quantitative trait loci (QTL)
in experimental populations.
The R/qtl2 software is a complete rewrite of the widely-used R/qtl software package.
The new software expands the scope of R/qtl to include
multi-parent populations derived from more than two founder strains, such as the
Collaborative Cross and Diversity Outbred mice, heterogeneous stocks, and
MAGIC plant populations. R/qtl2 is designed to handle modern high-density genotyping data
and high-dimensional molecular phenotypes including gene expression and proteomics.
R/qtl2 includes the ability to perform genome scans using a linear mixed model
to account for population structure, and also includes features to impute SNPs
based on founder strain genomes and to carry out association mapping.
The R/qtl2 software provides all of the basic features needed for QTL mapping,
including graphical displays and summary reports, and it can be extended
through the creation of add-on packages.
R/qtl2 comes with a test
framework and is free and open source software written in the R and
C++ programming languages.

\newpage

\section*{Introduction}

There has been a resurgence of interest in the mapping of quantitative
trait loci (QTL) in experimental organisms, spurred in part by the use
of gene expression phenotypes \citep[eQTL mapping; see][]{albert2015}
to more rapidly identify the underlying genes, and by the development
of multi-parent populations, including heterogeneous
stocks \citep{mott2000,mott2002}, MAGIC lines \citep{cavanagh2008, kover2009}, the Collaborative
Cross \citep{churchill2004}, and Diversity Outbred mice
\citep{churchill2012, svenson2012}.

There are numerous software packages for QTL mapping in experimental
populations, including Mapmaker/QTL \citep{lincoln1990}, QTL
Cartographer \citep{Basten2002}, R/qtl \citep{broman2003,
  broman_sen}, and MapQTL \citep{mapqtl}.
There are a smaller number of packages for QTL analysis in
multi-parent populations, including DOQTL \citep{gatti2014}, HAPPY
\citep{mott2000}, and mpMap \citep{mpMap}.

The original R/qtl (hereafter, R/qtl1) is widely used, and has a number of
advantages compared to proprietary alternatives. R/qtl1 includes a quite comprehensive set of QTL mapping
methods, including multiple-QTL exploration and model selection \citep{MQMpaper,broman2002,manichaikul2009}, as
well as extensive visualization
and data diagnostics tools \citep{broman_sen}. Further, users and developers both benefit by it
being an add-on package for the general statistical software, R
\citep{RCore}. A number of other R packages have been written to work in
concert with R/qtl1, including ASMap \citep{ASMap}, ctl \citep{ctl},
dlmap \citep{dlmap}, qtlcharts \citep{qtlcharts}, vqtl \citep{vqtl}, and
wgaim \citep{wgaim}.

R/qtl1 has a number of limitations
\citep[see][]{broman2014}, the most critical of which is that the
central data structure generally limits its use to biparental crosses.
Also, R/qtl1 was designed at a time when a dataset with more than 100 genetic
markers was considered large.

Rather than extend R/qtl1 for multi-parent populations, we decided to
start fresh. R/qtl2 is a completely redesigned R package for QTL
analysis that can handle a variety of multi-parent populations and is
suited for high-dimensional genotype and phenotype data. To
handle population structure, QTL analysis may be performed with a
linear mixed model that includes a residual polygenic effect. The
R/qtl2 software is available from its web site
(\url{https://kbroman.org/qtl2}) as well as GitHub
(\url{https://github.com/rqtl/qtl2}).


\clearpage
\section*{Features}

QTL analysis in multi-parent populations can be split into two parts:
calculation of genotype probabilities using multipoint SNP genotypes,
and the genome scan to evaluate the association between genotype and
phenotype, using those probabilities. As with R/qtl1, we use a hidden
Markov model \citep[HMM; see][App.\ D]{broman_sen} for the calculation
of genotype probabilities. As the source of genotype information, R/qtl2
considers SNP genotypes, generally derived from microarrays. At
present, we focus solely on marker genotypes rather than array
intensities or allele counts/dosages from genotyping-by-sequencing (GBS)
assays.

R/qtl2 includes implementations of the usual sorts of two-way crosses
(backcross, intercross, doubled haploids, two-way recombinant inbred
lines by selfing or sibling mating, and two-way advanced intercross
populations), and several different types of multi-parent populations
(4- and 8-way recombinant inbred lines by sibling mating; 4-, 8-, and
16-way recombinant inbred lines by selfing; 3-way advanced intercross
populations, Diversity Outbred mice, heterogeneous stocks, 19-way MAGIC
lines like the \citet{kover2009} Arabidopsis lines, and 6-way doubled
haploids following a design of maize MAGIC lines being developed at
the University of Wisconsin--Madison.

A key component of the HMM is the transition matrix (or ``step''
probabilities), which are specific to the cross design.
Transitions represent locations where the ancestry of chromosomal segments
change from one founder strain haplotype to another.
The transition probabilities
for multi-way recombinant inbred lines are taken from
\citet{broman2005}. The transition probabilities for heterogeneous
stocks and Diversity Outbred mice are taken from \citet{broman2012b}, which
uses the results of \citet{broman2012a}.

At present, R/qtl2 assumes dense marker information and a low level of
uncertainty in the haplotype reconstructions, so that we may rely on
Haley-Knott regression \citep{haley1992} for genome scans to establish
genotype-phenotype association. This may be performed either with a
simple linear model \citep[as in][]{haley1992}, or with a linear mixed
model \citep{yu2006, kang2008, lippert2011} that includes a residual
polygenic effect to account for population structure. The latter may
also be performed using kinship matrices derived using the
``leave-one-chromosome-out'' (LOCO) method \citep[see][]{yang2014}.

To establish statistical significance of evidence for QTL, accounting
for a genome scan, R/qtl2 facilitates the use of permutation tests
\citep{churchill1994}. For multi-parent populations with analysis via
a linear mixed model, we permute the rows of the haplotype reconstructions as
considered in \citet{cheng2013}.

R/qtl2 includes a variety of data diagnostic tools, which can be
particularly helpful for data on multi-parent populations where
the SNP genotypes are incompletely informative
(i.e., SNP genotypes do not fully define the corresponding founder haplotype).
These include SNP
genotyping error LOD scores \citep{Lincoln1992} and estimated
crossover counts.

R/qtl2 reproduces the functionality of DOQTL \citep{gatti2014} but
targets a broader set of multi-parent populations, in addition to
Diversity Outbred mice.
(DOQTL will ultimately be deprecated and replaced with R/qtl2.)
Fig.~1 contains a reproduction, using
R/qtl2, of Fig.~5 from \citet{gatti2014}. This is a QTL analysis of
constitutive neutrophil counts in 742 Diversity Outbred mice (from
generations 3--5) that were genotyped with the first generation Mouse
Universal Genotyping Array (MUGA) \citep{gigamuga}, which contained
7,851 markers, of which we are using 6,413.

\begin{figure}
  \includegraphics[width=\textwidth]{Figs/fig1.png}
  \caption{Reconstruction of Fig.~5 from \citet{gatti2014}, on the
    mapping of constitutive neutrophil counts in 742 Diversity Outbred
    mice. (A) LOD scores from a genome scan using the full model
    (comparing all 36 genotypes for the autosomes and 44 genotypes for
    the X chromosome); the dashed horizontal line indicates the 5\%
    genome-wide significance threshold, based on a permutation test.
    (B) LOD scores from a genome scan with an additive allele model
    (compare the 8 founder haplotypes). (C) LOD scores from a SNP
    association scan, using all SNPs that were genotyped in the eight
    founder lines. (D) Best linear unbiased predictors (BLUPs) of the
    eight haplotype effects in the additive model, along with the LOD
    curve on chromosome 1. (E) SNP association results in the region
    of the chromosome 1 QTL, along with genes in the region; SNPs with
    LOD scores within 1.5 of the maximum are highlighted in pink.
  All figures are produced with R/qtl2.}
\end{figure}

The regression model that R/qtl2 applies in a genome scan is determined by the
the HMM output in the genotype probabilities data structure.
For an 8-parent MPP such as the DO mice, there are 36 possible diplotypes
(44 on the X chromosome) and the genome scan will be based on a
regression model with 35 degrees of freedom.
With so many degrees of freedom, the model typically lacks power to detect QTL.
An alternative representation collapses the 36 states to 8 founder "dosages"
and uses a regression model with 7 degrees of freedom, assuming that the founder effects are additive at any given locus.
R/qtl2 has the ability to incorporate SNP (and other variant) data from founder strains and
to impute biallelic genotypes for every SNP.  The genome scan on imputed SNPs is
equivalent to an association mapping scan and can employ a additive
(one degree of freedom) or general (two degrees of freedom)
regression model.

Fig.~1A contains the LOD curves from a genome scan using a full model
comparing all 36 possible genotypes
with log neutrophil count as the phenotype and with sex and log white blood
cell count as covariates. The horizontal dashed line indicates the 5\%
genome-wide significance level, derived from a permutation test,
with separate thresholds for the autosomes and the X chromosome, using
the technique of \citet{broman2006}. Fig.~1B contains the LOD curves
from a genome scan using an additive allele model (corresponding to a
test with 7 degrees of freedom), and Fig.~1C contains a SNP
association scan, using a test with 2 degrees of freedom. All of these
analyses use a linear mixed model with kinship matrices derived using
the ``leave-one-chromosome-out'' (LOCO) method.

Fig.~1D shows the estimated QTL effects, assuming a single QTL with
additive allele effects on chr 1, and sliding the position of the QTL
across the chromosome. This is analogous to the estimated effects in
Fig.~5D of \citet{gatti2014}, but here we present Best Linear Unbiased
Predictors (BLUPs), taking the QTL effects to be random effects. This
results in estimated effects that have been shrunk towards 0, which
helps to clean up the figure and focus attention on the region of
interest.

Fig.~1E shows individual SNP association results, for the 6 Mbp region
on chr 1 that contains the QTL. As with the DOQTL software, we use all
available SNPs for which genotype data are available in the 8 founder
lines, and impute the SNP genotypes in the Diversity Outbred mice, using the
individuals' genotype probabilities along with the founder strains'
SNP genotypes.

Fig.~1 shows a number of differences from the results reported in
\citet{gatti2014}, including that we see nearly-significant loci on
chr 5 and 17 in the full model (Fig.~1A), and we see a second
significant QTL on chr 7 with the additive allele model (Fig.~1B).
Also, in Fig.~1E,  we see associated
SNPs not just at $\sim$128.6 Mbp near the \emph{Cxcr4\/} gene
\citep[as in][]{gatti2014}, but also a
group of associated SNPs at $\sim$ 132.4 Mbp, near \emph{Tmcc2}. The
differences between these results and those of \citet{gatti2014} are
largely due to differences in genotype probability calculations (data
not shown).

\subsection*{Data and software availability}

The data for Fig.~1 are available at the Mouse Phenotype Database
(\url{https://phenome.jax.org/projects/Gatti2}). In addition, R/qtl2
input files for these data are available at GitHub
(\url{https://github.com/rqtl/qtl2data}).

The R/qtl2 software is available from its web site
(\url{https://kbroman.org/qtl2}) as well as GitHub
(\url{https://github.com/rqtl/qtl2}). The software is licensed under
the GNU General Public License version 3.0.

The code to create Fig.~1 is available at GitHub at
\url{https://github.com/kbroman/Paper_Rqtl2}.

\clearpage
\section*{Implementation}

R/qtl2 is developed as a free and open source
add-on package for the general statistical software, R
\citep{RCore}. Much of the code is written in R, but computationally
intensive aspects are written in C++.  (Computationally intensive
aspects of R/qtl1 are in C.) We use Rcpp
\citep{eddelbuettel2011,Rcppbook} for the interface between R and C++,
to simplify code and reduce the need for copying data in memory.

Linear algebra calculations, such as matrix decomposition and linear
regression, are a central part of QTL analysis. We use
RcppEigen \citep{RcppEigen} and the Eigen C++ library \citep{eigen}
for these calculations. For the fit of linear mixed models, to account
for population structure with a residual polygenic effect, we closely
followed code from PyLMM \citep{pylmm}. In particular, we use the
basic technique described in \citet{kang2008}, of taking the eigen
decomposition of the kinship matrix.

We use roxygen2 \citep{roxygen2} to develop the documentation. R
packages make it easy to provide detailed documentation to users, but
the R documentation format is tedious and separate from the function
definitions and so requires the maintenance of two versions of
function documentation in separate places. Roxygen2 is a documentation
tool that uses specially-formatted comments near the original code to
generate the detailed R documentation files.

In contrast to R/qtl1, which includes almost no formal software tests,
R/qtl2 includes extensive unit tests to ensure correctness. We use the
R package testthat \citep{testthat} for this purpose. The use of unit
tests helps to catch bugs earlier, and has revealed several bugs in
R/qtl1.




\clearpage
\section*{Discussion}

R/qtl2 is a reimplementation of the R package R/qtl1, for QTL mapping
in multi-parent populations, with improved handling of
high-dimensional data. A variety of multi-parent populations have been
implemented, including heterogeneous stocks, the Collaborative Cross,
Diversity Outbred mice, and certain types of MAGIC lines. R/qtl2
includes the ability to perform genome scans via a linear mixed model,
accounting for population structure via a residual polygenic effect.

While the basic functionality of R/qtl2 is complete, there are a
number of areas for further development. In particular, we would like
to further expand the set of crosses that may be considered, including
partially-inbred recombinant inbred lines (so that we may deal with
residual heterozygosity, which presently is ignored). We have currently
been focusing on exact calculations for specific designs, but the
mathematics involved can be tedious. We would like to have a more
general approach for genotype reconstruction in multi-parent
populations, along the lines of RABBIT \citep{zheng2015} or STITCH
\citep{davies2016}. Plant researchers have been particularly creative
in developing unusual sets of MAGIC populations, and by our current
approach, each design requires the development of design-specific
code, which is difficult to sustain.

Another important area of development is the handling of
genotyping-by-sequencing (GBS) data. We are currently focusing solely
on called genotypes. With low-coverage GBS data, it is difficult
to get quality genotype calls at individual SNPs, and there will be
considerable advantage to using the pairs of allele counts and
combining information across SNPs. Extending the current HMM
implementation in R/qtl2 to handle pairs of allele counts for GBS data
appears straightforward.

At present, QTL analysis in R/qtl2 is solely by genome scans with
single-QTL models. Consideration of multiple-QTL models will be
particularly important for exploring the possibility of multiple
causal SNPs in a QTL region, along the lines of the CAVIAR software
\citep{caviar}.

Finally, we have currently focused solely on Haley-Knott regression
\citep{haley1992} for QTL analysis. This has a big advantage in terms
of computational speed, but it does not fully account for the
uncertainty in genotype reconstructions. But the QTL analysis
literature has a long history of methods for dealing with this
genotype uncertainty, including
interval mapping \citep{lander1989} and
multiple imputation \citep{sen2001}. While this has not been a high
priority in the development of R/qtl2, ultimately we will include
implementations of these sorts of approaches, to better handle regions
with reduced genotype information.

\clearpage
\section*{Acknowledgments}

This work was supported in part by National Institutes of Health
grants R01GM074244 (to K.W.B.), R01GM070683 (to K.W.B. and G.A.C.),
and R01GM123489 (to \'S.S.).


\clearpage
\bibliographystyle{genetics}
\renewcommand*{\refname}{\normalfont\sffamily\bfseries Literature Cited}
\bibliography{rqtl2_paper}


\end{document}
